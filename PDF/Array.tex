\chapter{Αταξινόμητος πίνακας}

\section{Εισαγωγή}

Ο αταξινόμητος πίνακας αποτελεί την χειρότερη δομή δεδομένων για την διαχείριση μεγάλου όγκου δεδομένων. Η εισαγωγή των δεδομένων (λέξεων) γίνετε γραμμικά (δηλ. κάθε λέξη ακολουθεί εισάγεται πίσω από την προηγουμένη). Η αναζήτηση πραγματοποιείται σειριακά (δηλ. εξετάζεται κάθε λέξη μέχρι να βρεθεί ή όχι). Παρόμοια με την αναζήτηση, πραγματοποιείται και η διαγραφή κάποιας λέξης από τον πίνακα.

\section{Λειτουργίες}

\subsection{Εισαγωγή}

Η συνάρτηση αρχικά εξετάζει εάν υπάρχει ήδη η λέξη που πρόκειται να εισαχθεί, για αυτό τον έλεγχο καλείται η αναζήτηση που εντοπίζει την θέση της λέξης και την αποθηκεύει σε μια μεταβλητή. Στην περίπτωση που υπάρχει ήδη, η λέξη δεν εισάγεται ξανά στον πίνακα αλλά αυξάνεται ο αριθμός των εμφανίσεων της λέξης αυτής. Αντίθετα όταν δεν υπάρχει εισάγεται στον πίνακα θέτοντας παράλληλα των αριθμό των εμφανίσεων της λέξης σε ένα (1).

\subsection{Αναζήτηση}

Στην αρχή γίνεται η αρχικοποίηση της μεταβλητής που θα αποθηκεύσει την θέση της λέξης που αναζητείται όπου θα την επιστρέψει κιόλας. Η διαδικασία που πραγματοποιείται έχει ως εξής: κάθε στοιχείο του πίνακα εξετάζεται μέχρι να βρεθεί το ζητούμενο ή δεν υπάρχουν αλλά στοιχεία να προσπελαστούν. Στην τελευταία περίπτωση, θα επιστραφεί η τιμή μείον ένα (-1) από την συνάρτηση ενώ σε κάθε άλλη περίπτωση θα επιστραφεί η θέση της λέξης μέσα στον πίνακα. 

\subsection{Διαγραφή}

Αρχικά γίνεται κλήση της συνάρτησης αναζήτησης ώστε να βρεθεί η θέση της λέξης που πρόκειται να διαγραφεί. Σε περίπτωση που η λέξη εμφανίζεται περισσότερες από μια φορές, το πεδίο του struct που αποθηκεύει τον αριθμό των εμφανίσεων της λέξης. Εάν η τιμή του πεδίου πριν την κλήση της συνάρτησης είναι ένα (1) τότε, αφού κληθεί αφαιρείται η λέξη από τον πίνακα, τα στοιχεία μετακινούνται κατά μια θέση προς τα αριστερά και το μέγεθος του πίνακα μειώνεται κατά ένα.
